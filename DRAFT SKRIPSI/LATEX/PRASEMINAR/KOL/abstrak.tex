%----------------------------------------------------------------------------------------
%	ABSTRACT
%----------------------------------------------------------------------------------------
\Abstract{\scriptsize 
% ---- Tuliskan abstrak di bagian ini seperti contoh.
Indonesia memiliki lebih dari 32.000 spesies tumbuhan. Saat ini hutan Indonesia mengalami kerusakan dan kepunahan. Oleh karena itu, diperlukan upaya untuk melestarikan tumbuhan. Salah satu cara untuk melestarikan tumbuhan adalah dengan cara mengenali spesies tumbuhan tersebut.   Berdasarkan hal tersebut maka muncul bidang baru dalam pengumpulan informasi tumbuhan yang bernama \textit{biodiversity informatics}. Metode pemodelan data yang dapat menangani sistem berbasis inferensi adalah ontologi. Ontologi dapat diterapkan pada web semantik. Penelitian ini akan mengembangkan sistem web semantik yang memberikan informasi genetika tumbuhan. Selain itu juga merumuskan masalah bagaimana melakukan inferensi pengetahuan dari sistem web semantik gen tumbuhan dengan sistem web semantik yang lain.
% ---- Akhir bagian abstrak
\normalsize}
